\documentclass[a4paper,oneside]{ctexart}

\usepackage{amsmath,amsthm}
\usepackage{hyperref}
\usepackage{graphicx}
\usepackage{ifthen}
\usepackage{titlesec}
%\usepackage{verbatim}
%\usepackage{fancyvrb}
%\usepackage{xcolor}
%\usepackage{listings}
%\usepackage{bm}

\hypersetup{linkcolor=black,
            citecolor=black,
            linkcolor=black}

\usepackage{fancyhdr}
\usepackage{lastpage}
%\usepackage{booktabs}
%\usepackage{longtable}


%\usepackage[numbers,super,square,sort&compress]{natbib}
%\bibliographystyle{unsrt}


\begin{document}

\title{\zihao{2}\bfseries 豆瓣图书L推荐系统文档}
\author{Locke}
\maketitle

\section{数据来源}

作为一个推荐系统,必然需要数据源,而且得是一个足够大的数据源,
所以如何获得这些数据就是第一个问题。

按照传统思路,逐个抓取数据是最直接的办法,但是很遗憾,豆瓣并不是第一天存在了,
也和各种爬虫斗争了这么长时间,相信它的“反数据挖据型爬虫”已经做得很好了,
所以要想获得足够多的数据往往需要相当长的时间,这是很不利的。
另一个问题是,豆瓣上有相当多的不活跃用户,很多用户的评分信息为0,
这进一步降低了直接爬虫的收益。
所以直接从豆瓣上爬数据只能作为补充方案,作为主要数据来源并不适合。

无奈之余偶然在网络上发现了一个豆瓣评分的数据包:
\url{http://www.datatang.com/data/42832/},
里面大概有383K个用户的3.6M条评分数据,这里要向共享这个数据的童鞋表示衷心的感谢。
这个作为主要数据来源应该够了。
但其中的数据就只有评分,忽略了是否有评论、tag、无评分记录、时间等信息。

然后才意识到搜索引擎这个爬虫的老大哥那也应该有页面的数据,
经过尝试,Google上有大量豆瓣图书页面的缓存,
缓存的URL格式如
\begin{quotation}
http://webcache.googleusercontent.com/search?q=

\quad cache:book.douban.com/subject/2152385/
\end{quotation}
但对于用户信息页面的则相对较少,可以作为补充。
百度的快照暂时还没有看出规律,不过可以通过在baidu搜索URL来获得快照地址。

所以总的数据来源是:主要从数据包获取,书籍信息通过搜索引擎快照,其他信息通过从豆瓣网站直接抓取。

\end{document}
